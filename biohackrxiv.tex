\documentclass[a4paper,10pt]{article}
\usepackage[english]{babel}
\usepackage[utf8]{inputenc}

\usepackage{graphicx}

\usepackage{fancyhdr}
% Turn on the style
\pagestyle{fancyplain}
% Clear the header and footer
\fancyhead{}
\fancyfoot{}
% Add header
\lhead{\begin{picture}(0,0) \put(0,0){\includegraphics[width=4cm]{./biohackrxiv.png}} \end{picture}}
\fancyheadoffset[L]{\dimexpr\oddsidemargin+1in\relax}
% Add footer
\rfoot{\thepage}

\usepackage{lipsum}

\usepackage{parskip}

\usepackage{authblk}

\title{BioHackrXiv contribution template}
\author[1,2]{Leyla Garcia (0000-0003-3986-0510)}
\author[1,3]{Alexander Garcia (0000-0003-1238-2539)}
\author[1,4]{Pjotr Prins (0000-0002-8021-9162)}
\author[1,5]{Toshiaki Katayama (0000-0003-2391-0384)}
\affil[1]{BioHackrXiv, http://biohackrxiv.org}
\affil[2]{ZB MED Information Centre for Life Sciences, Gleueler Str. 60, 50931 Cologne, Germany}
\affil[3]{BASF, Carl-Bosch-Strasse 38, 67056 Ludwigshafen am Rhein, Germany}
\affil[4]{University of Tennessee Health Science Center, Memphis, TN, US}
\affil[5]{Database Center for Life Science. 178-4-4 Wakashiba, Kashiwa-shi, Chiba 277-0871, Japan}
\date{}

\setcounter{secnumdepth}{-2}
\begin{document}
\maketitle

\section{Introduction or Background}\label{introduction-or-background}

BioHackrXiv is a preprint to report on works done during BioHackathons, CodeFests, Sprints, VoCamps or similar events and related to Life Sciences and Health Care domains. Articles in BioHackrXiv commonly report on-going work as a couple of days of hacking are commonly not enough to get the things fully done. However, hackathon reports should still show work
that people can build upon.

One of the reasons behind BioHackathons and similar is sharing: ideas, data, designs, software, documentations and tutorials among others. We therefore kindly ask you to use CC-BY 4.0 license for your work \cite{creative_commons_cc-by_nodate}. We also encourage you to share your code on GitHub, with an open license whenever possible.

The people behind BioHackrXiv has participated on multiple BioHackathons organized by National Bioscience Database Center (NBDC) / Database Center for Life Science (DBCLS) in Japan and ELIXIR Europe. We decided to create BioHackrXiv preprint in order to make it easier for all to share and report on the work done at this sort of events. Of course, any hacking event is welcome to publish here, the more the merrier they say! And, if you want to go for a more formal peer-reviewed publication, you can always do so once your work is more mature \cite{katayama_dbcls_2010}.

Add to this section a couple of paragraphs introducing the work done where you (partially) develop the work you are reporting here.

Our templates use styles adding some space after paragraphs rather than indentation.

\subsection{Subsection level 2}\label{subsection-level-2}

Please keep sections to a maximum of three levels, even better if only two levels.

\subsubsection{Subsection level 3}\label{subsection-level-3}

Please keep sections to a maximum of three levels.

\subsection{Where to find templates to submit to
BioHackrXiv}\label{where-to-find-templates-to-submit-to-biohackrxiv}

Our Word template is loosely inspired on the Lecture Notes in Computer Science (LNCS) templates \cite{lncs_springer_conference_nodate}. We have also looked at the Journal of Open Source Software submission process \cite{joss_submitting_2017}; in
particular, we offer a PDF generation from Markdown based on theirs.

We provide a Word template and a Markdown template on our GitHub repository. If you use the Word template, please export it or print it as PDF. If you use the Markdown template, you should use the generator tool \cite{joss_whedon_nodate}. Please have a look to the instructions page for BioHackrXiv \cite{prins_submitting_2020}.

\subsection{Tables, figures and so on}\label{tables-figures-and-so-on}

Please remember to introduce tables (see Table 1) before they appear on the document. We recommend to center tables, formulas and figure but not the corresponding captions. Feel free to modify the table style as it better suits to your data.


\begin{center}
 \begin{tabular}{c c c } 
 \hline\hline
 Heading level & Description & Font size and style \\
 \hline
 Title	& Article title & 18 point, bold \\
 1st-level heading & First level subtitle & 14 point, bold \\
 2nd-level heading & Second level subtitle & 12 point, bold \\
 3rd-level heading & Third level subtitle & 10 point, bold \\
 Normal & Normal text & 10 point \\
 \hline
\end{tabular}
\end{center}



Remember to introduce figures (see Figure 1) before they appear on the document.

\begin{figure}
\centering
\includegraphics{./biohackrxiv.png}
\caption{BioHackrXiv}
\end{figure}

Figure 1. A figure corresponding to the logo of our BioHackrXiv preprint.

\section{Other main section on your manuscript level
1}\label{other-main-section-on-your-manuscript-level-1}

Feel free to use numbered lists or bullet points as you need. 
\begin{itemize}
  \item Item 1
  \item Item 2
\end{itemize}

\section{Discussion and/or
Conclusion}\label{discussion-andor-conclusion}

We recommend to include some discussion or conclusion about your work. Feel free to modify the section title as it fits better to your manuscript.

\section{Future work}\label{future-work}

And maybe you want to add a sentence or two on how you plan to continue. Please keep reading to learn about citations and references.

For citations of references, we prefer the use of parenthesis, last name and year. If you use a citation manager, Elsevier -- Harvard or American Psychological Association (APA) will work. If you are referencing web pages, software or so, please do so in the same way. Whenever possible, add authors and year. We have included a couple of citations along this document for you to get the idea. Please remember to always add DOI whenever available, if not possible, please provide alternative URLs. You will end up with an alphabetical order list by authors' last name.

\section{Jupyter notebooks, GitHub repositories and data
repositories}\label{jupyter-notebooks-github-repositories-and-data-repositories}

\begin{itemize}
\item
  Please add a list here
\item
  Make sure you let us know which of these correspond to Jupyter
  notebooks. Although not supported yet, we plan to add features for
  them
\item
  And remember, software and data need a license for them to be used by
  others, no license means no clear rules so nobody could legally use a
  non-licensed research object, whatever that object is
\item
  Jupyter notebook, name, link, license
\item
  Github repo, name, link, license
\item
  Data repo, name, link, license
\end{itemize}

\section{Acknowledgements}\label{Acknowledgements}
Please always remember to acknowledge the BioHackathon, CodeFest, Sprint, VoCamp where you participated and (partially) developed the work you are reporting here. Remember to also include the link to the hacking event here.

\medskip

\bibliographystyle{apalike}
\bibliography{biohackrxiv}

\end{document}

